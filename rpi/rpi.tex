\documentclass{beamer}

\usepackage[portuguese]{babel}
\usepackage[utf8]{inputenc}
\usepackage{hyperref}
\usepackage[style=brazilian]{csquotes}
\usepackage{subcaption}
\usepackage{xcolor}
\usepackage{minted}
\usepackage{multicol}

\def\signed #1{{\leavevmode\unskip\nobreak\hfil\penalty50\hskip1em
  \hbox{}\nobreak\hfill #1%
  \parfillskip=0pt \finalhyphendemerits=0 \endgraf}}

\newsavebox\mybox
\newenvironment{aquote}[1]
  {\savebox\mybox{#1}\begin{quote}\openautoquote\hspace*{-.7ex}}
  {\unskip\closeautoquote\vspace*{1mm}\signed{\usebox\mybox}\end{quote}}

\usetheme{boxes}

\title{Lego Mindstorm + Raspberry Pi}
\date{}
\institute{\small MAC0318 - Introdução à Programação de Robôs Móveis\\~\\\scriptsize Instituto de
Matemática e Estatística (IME)\\Universidade de São Paulo (USP)}

\begin{document}

\begin{frame}
  \titlepage
\end{frame}

\begin{frame}[fragile]
  \frametitle{Acessando o Raspberry Pi}

  Raspberry Pi's estão cadastradas nas faixas de IP de \texttt{172.26.0.10} até
  \texttt{172.26.0.13}.
  \vspace{1cm}

  \begin{minted}{bash}
    ssh pi@172.26.0.10
  \end{minted}
  \vspace{1cm}

  Os computadores devem estar conectados à \texttt{eduroam} ou \texttt{Rede IME}, enquanto que as
  placas devem estar conectadas à Rede \texttt{MAC0318}.
\end{frame}

\begin{frame}[fragile]
  \frametitle{Descobrindo o IP da sua Raspberry Pi}

  \begin{minted}{bash}
    hostname -I
  \end{minted}
\end{frame}

\begin{frame}[fragile]
  \frametitle{Passando arquivos por SSH}

  Para passar arquivos do computador ao Raspberry Pi:
  \vspace{1cm}

  \begin{minted}{bash}
scp caminho/de/origem pi@172.26.0.10:caminho/de/destino
scp ~/mac0318/code.py pi@172.26.0.10:~/
  \end{minted}
  \vspace{1cm}

  Para passar arquivos do Raspberry Pi ao computador:
  \vspace{1cm}

  \begin{minted}{bash}
scp pi@172.26.0.10:caminho/de/origem ~/caminho/de/destino
scp pi@172.26.0.10:~/code.py ~/mac0318/
  \end{minted}
\end{frame}

\begin{frame}[fragile]
  \frametitle{Passando arquivos grandes}

  Se os arquivos forem muito grandes, muitas vezes é mais rápido conectar o SD da placa direto no
  computador por meio do adaptador micro-SD para SD.
\end{frame}

\end{document}
